% Copyright 2019 by Till Tantau and others Wibrow
%
% This file may be distributed and/or modified
%
% 1. under the LaTeX Project Public License and/or
% 2. under the GNU Public License.
%
% See the file doc/generic/pgf/licenses/LICENSE for more details.

\usetikzlibrary{calc,decorations.markings}%


%
% Indicate that the picture contains a circuit
%
\tikzset{
  circuit/.style={
    every circuit/.try,
    /utils/exec=\tikz@lib@circ@on@tofalse,
    execute at begin to={
      \tikz@lib@circ@on@totrue
      \let\tikz@lib@circ@start@node\pgfutil@empty
      \def\tikz@lib@circ@end{-- (\tikztotarget) \tikztonodes}
    }
  }
}%

\newif\iftikz@lib@circ@on@to


%
% General management
%

\tikzset{
  circuit handle symbol/.code={%
    \global\advance\tikz@lib@circ@count by1\relax%
    \iftikz@lib@circ@on@to
      {%
        % Compute the tikz@time...
        \pgfkeys{
          /pgf/key filters/active families or no family/.install key filter={/pgf/key filters/false}{/pgf/key filters/false},
          /tikz/circuits/pos grabber/.activate family}
        \pgfqkeysfiltered{/tikz}{pos=.5,#1}%
        \expandafter%
      }%
      \expandafter\def\expandafter\tikz@lib@circ@pos\expandafter{\tikz@time}%
      \pgfkeysalso{to path={
         \pgfextra{\tikz@lib@circ@on@tofalse}
         decorate [decoration={markings,mark connection node=mark node}]
          {
            \tikz@lib@circ@start@node
            \tikz@lib@circ@end
          }
        }}
      \ifdim\tikz@lib@circ@pos pt=0pt\relax%
        \def\tikz@lib@circ@start@node{%
          {\pgfextra{\tikz@lib@circ@compute@direction{\tikz@lib@circ@start}}%
          node[alias=tikz@lib@circ@node@start,at={(0,0)},#1,pos=]{}}(tikz@lib@circ@node@start)}%
      \else%
        \ifdim\tikz@lib@circ@pos pt=1pt\relax%
          \def\tikz@lib@circ@end{
            {\pgfextra{\tikz@lib@circ@compute@direction{\tikz@lib@circ@target}}%
              node[alias=tikz@lib@circ@node\the\tikz@lib@circ@count,at={(0,0)},#1,pos=]{}}
            --(tikz@lib@circ@node\the\tikz@lib@circ@count)\tikztonodes
          }%
        \else%
          \edef\tikz@marshal{mark=at position \tikz@lib@circ@pos\space with}%
          \def\tikz@marshala{decoration=}%
          \expandafter\expandafter\expandafter\tikzset%
          \expandafter\expandafter\expandafter{\expandafter\tikz@marshala\expandafter{\tikz@marshal{\node[alias=mark node,#1,pos=]{};}}}
        \fi%
      \fi%
    \else
      \pgfkeysalso{#1}
    \fi
  }
}%
\newcount\tikz@lib@circ@count

\def\tikz@lib@circ@compute@direction#1{%
  \tikz@scan@one@point\tikz@lib@circ@save@start(\tikztostart)%
  \tikz@scan@one@point\tikz@lib@circ@save@target(\tikztotarget)%
  \pgf@process{\pgfpointnormalised{\pgfpointdiff{\tikz@lib@circ@start}{\tikz@lib@circ@target}}}%
  \pgf@ya=-\pgf@y%
  \pgftransformcm{\the\pgf@x}{\the\pgf@y}{\the\pgf@ya}{\the\pgf@x}{#1}%
}%
\def\tikz@lib@circ@save@start#1{\def\tikz@lib@circ@start{#1}}%
\def\tikz@lib@circ@save@target#1{\def\tikz@lib@circ@target{#1}}%

\tikzset{
  circuits/pos grabber/.is family,
  pos/.belongs to family=/tikz/circuits/pos grabber,
  at start/.belongs to family=/tikz/circuits/pos grabber,
  very near start/.belongs to family=/tikz/circuits/pos grabber,
  near start/.belongs to family=/tikz/circuits/pos grabber,
  midway/.belongs to family=/tikz/circuits/pos grabber,
  near end/.belongs to family=/tikz/circuits/pos grabber,
  very near end/.belongs to family=/tikz/circuits/pos grabber,
  at end/.belongs to family=/tikz/circuits/pos grabber
}%


%
% Declaring a new symbol
%
\tikzset{
  circuit declare symbol/.style args={#1}{
    % Declares a new circuit symbol.
    %
    % #1 = name
    % #2 = factor for default minimum width
    % #3 = factor for default minimum height
    %
    % The following styles are defined:
    %
    % #1
    % Can be given as an option to a "node" command or to a "to"
    % command. It takes a set of options as parameter that will be
    % passed to the option.
    %
    % set #1 graphic
    % This keys can be set to the necessary options to make a normal
    % node look like the symbol. For instance, for a resistor that is
    % rendered as a rectangle, this keys can be set, basically, to
    % "rectangle,draw". Set this key to use a different appearance for
    % symbols of kind #1.
    %
    % every #1
    % This style will be included with every symbols of this kind and
    % can be used to configure them further.
    %
    % every circuit symbol
    % This style is also included with every symbol.
    #1/.style={circuit handle symbol={
        inner sep=0.5pt,
        every circuit symbol,
        #1/graphic,
        every #1/.try,
        ##1}
    },
    #1/graphic/.style={},
    set #1 graphic/.style={#1/graphic/.style={##1}}
  },
  circuit symbol unit/.code=\pgfmathsetlength\tikzcircuitssizeunit{#1},
  circuit symbol size/.style args={width #1 height #2}{
    minimum width={(#1)*\tikzcircuitssizeunit},
    minimum height={(#2)*\tikzcircuitssizeunit}
  },
  huge circuit symbols/.style={circuit symbol unit=10pt},
  large circuit symbols/.style={circuit symbol unit=8pt},
  medium circuit symbols/.style={circuit symbol unit=7pt},
  small circuit symbols/.style={circuit symbol unit=6pt},
  tiny circuit symbols/.style={circuit symbol unit=5pt},
}%

\newdimen\tikzcircuitssizeunit
\tikzcircuitssizeunit=7pt


%
% Annotations
%

\tikzset{
  circuit declare annotation/.style args={#1#2#3}{%
    #1/.style={
      append after command={%
        \bgroup
          [current point is local=true]
          [every circuit annotation/.try]
          [every #1/.try]
          [shift={(\tikzlastnode.north)}]
          [annotation arrow,->]
          [label distance=#2,##1]
          #3
          \tikz@after@path%
        \egroup%
      }
    },
    #1'/.style={
      append after command={%
        \bgroup
          [current point is local=true]
          [every circuit annotation/.try]
          [every #1/.try]
          [shift={(\tikzlastnode.south)},yscale=-1]
          [annotation arrow,->]
          [label distance=#2,##1]
          #3
          \tikz@after@path%
        \egroup%
      }
    }
  },
  annotation arrow/.style = {
    /utils/exec={\pgfsetarrowoptions{direction ee}{.4*\the\tikzcircuitssizeunit+.3*\the\pgflinewidth}},
    >=direction ee
  }
}%



%
% Rotating symbols
%
\tikzset{
  point up/.style={rotate=90},
  point down/.style={rotate=-90},
  point left/.style={rotate=180},
  point right/.style={}
}%



%
% Basic theming
%

\tikzset{
  every circuit symbol/.style={},
  circuit symbol open/.style={draw},
  circuit symbol filled/.style={draw,fill=black},
  circuit symbol lines/.style={draw},
  circuit symbol wires/.style={draw},
}%





%
% Labels
%

\tikzset{
  info/.code={\pgfutil@ifnextchar[\tikz@lib@circ@lab@plain{\tikz@lib@circ@lab@plain[]}#1\pgf@stop},%}
  info'/.code={\pgfutil@ifnextchar[\tikz@lib@circ@labp@plain{\tikz@lib@circ@labp@plain[]}#1\pgf@stop},%}
  info sloped/.code={\pgfutil@ifnextchar[\tikz@lib@circ@lab@sloped@plain{\tikz@lib@circ@lab@sloped@plain[]}#1\pgf@stop},%}
  info' sloped/.code={\pgfutil@ifnextchar[\tikz@lib@circ@lab@slopedp@plain{\tikz@lib@circ@lab@slopedp@plain[]}#1\pgf@stop},%}
  circuit declare unit/.style 2 args={
    %
    % Defines four styles that can be used to add labels to a node.
    %
    #1/.code={\pgfutil@ifnextchar[\tikz@lib@circ@lab{\tikz@lib@circ@lab[]}##1\pgf@stop{#2}{#1}},%}
    #1 sloped/.code={\pgfutil@ifnextchar[\tikz@lib@circ@lab@sloped{\tikz@lib@circ@lab@sloped[]}##1\pgf@stop{#2}{#1}},%}
    #1'/.code={\pgfutil@ifnextchar[\tikz@lib@circ@labp{\tikz@lib@circ@labp[]}##1\pgf@stop{#2}{#1}},%}
    #1' sloped/.code={\pgfutil@ifnextchar[\tikz@lib@circ@lab@slopedp{\tikz@lib@circ@lab@slopedp[]}##1\pgf@stop{#2}{#1}}%}
  }
}%

\def\tikz@lib@circ@lab[#1]#2\pgf@stop#3#4{\tikzset{label={[every info/.try,every #4/.try,#1]$\mathrm{#2#3}$}}}%
\def\tikz@lib@circ@lab@sloped[#1]#2\pgf@stop#3#4{\tikzset{label={[every info/.try,every #4/.try,transform shape,#1]$\mathrm{#2#3}$}}}%
\def\tikz@lib@circ@labp[#1]#2\pgf@stop#3#4{\tikzset{label={[label position=below,every info/.try,every #4/.try,#1]$\mathrm{#2#3}$}}}%
\def\tikz@lib@circ@lab@slopedp[#1]#2\pgf@stop#3#4{\tikzset{label={[label position=below,every info/.try,every #4/.try,transform shape,#1]$\mathrm{#2#3}$}}}%

\def\tikz@lib@circ@lab@plain[#1]#2\pgf@stop{\tikzset{label={[every info/.try,#1]#2}}}%
\def\tikz@lib@circ@lab@sloped@plain[#1]#2\pgf@stop{\tikzset{label={[every info/.try,transform shape,#1]#2}}}%
\def\tikz@lib@circ@labp@plain[#1]#2\pgf@stop{\tikzset{label={[label position=below,every info/.try,#1]#2}}}%
\def\tikz@lib@circ@lab@slopedp@plain[#1]#2\pgf@stop{\tikzset{label={[label position=below,every info/.try,transform shape,#1]#2}}}%



\endinput
